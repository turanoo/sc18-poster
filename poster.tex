% The MIT License (MIT)
% =====================

% **Copyright (c) 2018 Anish Athalye (me@anishathalye.com)**

% Permission is hereby granted, free of charge, to any person obtaining a copy of
% this software and associated documentation files (the "Software"), to deal in
% the Software without restriction, including without limitation the rights to
% use, copy, modify, merge, publish, distribute, sublicense, and/or sell copies
% of the Software, and to permit persons to whom the Software is furnished to do
% so, subject to the following conditions:

% The above copyright notice and this permission notice shall be included in all
% copies or substantial portions of the Software.

% THE SOFTWARE IS PROVIDED "AS IS", WITHOUT WARRANTY OF ANY KIND, EXPRESS OR
% IMPLIED, INCLUDING BUT NOT LIMITED TO THE WARRANTIES OF MERCHANTABILITY,
% FITNESS FOR A PARTICULAR PURPOSE AND NONINFRINGEMENT. IN NO EVENT SHALL THE
% AUTHORS OR COPYRIGHT HOLDERS BE LIABLE FOR ANY CLAIM, DAMAGES OR OTHER
% LIABILITY, WHETHER IN AN ACTION OF CONTRACT, TORT OR OTHERWISE, ARISING FROM,
% OUT OF OR IN CONNECTION WITH THE SOFTWARE OR THE USE OR OTHER DEALINGS IN THE
% SOFTWARE.


% Gemini theme
% https://github.com/anishathalye/gemini

\documentclass[final]{beamer}

% ====================
% Packages
% ====================

\usepackage[T1]{fontenc}
\usepackage{lmodern}
\usepackage[size=custom,width=120,height=72,scale=1.0]{beamerposter}
\usetheme{gemini}
\usecolortheme{gemini}
\usepackage{graphicx}
\usepackage{booktabs}
\usepackage{tikz}
\usepackage{pgfplots}

% ====================
% Lengths
% ====================

% If you have N columns, choose \sepwidth and \colwidth such that
% (N+1)*\sepwidth + N*\colwidth = \paperwidth
\newlength{\sepwidth}
\newlength{\colwidth}
\setlength{\sepwidth}{0.025\paperwidth}
\setlength{\colwidth}{0.3\paperwidth}

\newcommand{\separatorcolumn}{\begin{column}{\sepwidth}\end{column}}

% ====================
% Title
% ====================

\title{Enabling Reproducible Microbiome Science through Decentralized Provenance Tracking in QIIME 2}

\author{Chris Keefe \inst{1} \and Turan Naimey}

\institute[shortinst]{\inst{1} Northern Arizona University; Pathogen and Microbiome Institute}

% ====================
% Body
% ====================

\begin{document}

\begin{frame}[t]
\begin{columns}[t]
\separatorcolumn

\begin{column}{\colwidth}

  \begin{block}{Objective and Introduction}

    \textbf{Objective:} To demonstrate the ways in which automatic, integrated, decentralized
    provenance tracking in QIIME 2 enables reproducible microbiome science,
    using a sample analysis as framework for discussion.

    \textbf{Introduction:} QIIME 2 is a powerful, extensible, and decentralized microbiome analysis
    package with a focus on data and analysis transparency. QIIME 2 allows
    researchers to start an analysis with raw DNA sequence data and finish with
    publication-quality figures and statistical results.

    \textbf{QIIME 2 Key Features:}
    \begin{itemize}
      \item Integrated and automatic tracking of data provenance
      \item Semantic type system
      \item Plugin system for extending microbiome analysis functionality
      \item Support for multiple types of user interfaces (e.g. API, command line, graphical)
    \end{itemize}
  \end{block}

  \begin{block}{Creating a QIIME 2 artifact = initiating provenance tracking}

   In order to work in QIIME 2, we must first import the data we plan to
   analyze and prepare it for analysis.  We are working with multiplexed
   paired-end sequences, so we’ll place the necessary files into a new
   directory, and run \code{qiime tools import <args>} to create a new QIIME 2 artifact.

   \textbf{A QIIME 2  Artifact is an immutable collection of data and its associated
   metadata, including the semantic type of the data, its format, and its
   provenance - that is, how the data was generated}.

   \textbf{Using Artifacts rather than raw data allows QIIME 2 to ensure actions are only
   performed on appropriate data types}.

   \textbf{Each action generates new artifacts whose provenance metadata includes all
   actions since (and including) the original import. Further, it will contain
   a comprehensive history of the data in the artifact, including:}

    \begin{itemize}
      \item \textbf{the processes to which the data was subjected}
      \item \textbf{all parameters chosen when running these processes}
      \item \textbf{citation information relevant to the methods chosen}
      \item \textbf{version information for all software involved (QIIME 2 and all relevant plugins)}
    \end{itemize}

    Before we begin our analysis, it is necessary for us to demultiplex our
    data. We will pass our newly-created artifact into the \code{q2-demux plugin},
    alongside a column of sample metadata containing barcode sequences. Working
    with data like these frequently, we scripted a simple command-line workflow
    for a shell to handle these first steps quickly and automatically. \code{link to
    source repo}

  \end{block}

  \begin{block}{QIIME 2 analyses are host-agnostic. Provenance metadata storage is decentralized.}

    The next phase in our analysis, de-noising the data, will be computationally
    intensive. A multi-day runtime on an HPC node is typical, so we will take
    advantage of a high-performance compute cluster for our analysis.

    \code{[Screencap: CLI running on Monsoon - sub workstation CLI for now]}

    \textbf{The integrated and decentralized handling of provenance metadata in
    artifacts makes it easy for QIIME 2 users to transfer files and run jobs
    through any QIIME 2 interface on any host, accumulating and retaining
    accurate accurate provenance information}.

    By incorporating this information and the relevant data into a single
    artifact file, that file can be shared, viewed, and manipulated easily
    without risk of losing or mis-recording the methods used.

  \end{block}

\end{column}

\separatorcolumn

\begin{column}{\colwidth}

  \begin{block}{A block containing an enumerated list}

    Vivamus congue volutpat elit non semper. Praesent molestie nec erat ac
    interdum. In quis suscipit erat. \textbf{Phasellus mauris felis, molestie
    ac pharetra quis}, tempus nec ante. Donec finibus ante vel purus mollis
    fermentum. Sed felis mi, pharetra eget nibh a, feugiat eleifend dolor. Nam
    mollis condimentum purus quis sodales. Nullam eu felis eu nulla eleifend
    bibendum nec eu lorem. Vivamus felis velit, volutpat ut facilisis ac,
    commodo in metus.

    \begin{enumerate}
      \item \textbf{Morbi mauris purus}, egestas at vehicula et, convallis
        accumsan orci. Orci varius natoque penatibus et magnis dis parturient
        montes, nascetur ridiculus mus.
      \item \textbf{Cras vehicula blandit urna ut maximus}. Aliquam blandit nec
        massa ac sollicitudin. Curabitur cursus, metus nec imperdiet bibendum,
        velit lectus faucibus dolor, quis gravida metus mauris gravida turpis.
      \item \textbf{Vestibulum et massa diam}. Phasellus fermentum augue non
        nulla accumsan, non rhoncus lectus condimentum.
    \end{enumerate}

  \end{block}

  \begin{block}{Fusce aliquam magna velit}

    Et rutrum ex euismod vel. Pellentesque ultricies, velit in fermentum
    vestibulum, lectus nisi pretium nibh, sit amet aliquam lectus augue vel
    velit. Suspendisse rhoncus massa porttitor augue feugiat molestie. Sed
    molestie ut orci nec malesuada. Sed ultricies feugiat est fringilla
    posuere.

    \begin{figure}
      \centering
      \begin{tikzpicture}
        \begin{axis}[
            scale only axis,
            no markers,
            domain=0:2*pi,
            samples=100,
            axis lines=center,
            axis line style={-},
            ticks=none]
          \addplot[red] {sin(deg(x))};
          \addplot[blue] {cos(deg(x))};
        \end{axis}
      \end{tikzpicture}
      \caption{Another figure caption.}
    \end{figure}

  \end{block}

  \begin{block}{Nam cursus consequat egestas}

    Nulla eget sem quam. Ut aliquam volutpat nisi vestibulum convallis. Nunc a
    lectus et eros facilisis hendrerit eu non urna. Interdum et malesuada fames
    ac ante \textit{ipsum primis} in faucibus. Etiam sit amet velit eget sem
    euismod tristique. Praesent enim erat, porta vel mattis sed, pharetra sed
    ipsum. Morbi commodo condimentum massa, \textit{tempus venenatis} massa
    hendrerit quis. Maecenas sed porta est. Praesent mollis interdum lectus,
    sit amet sollicitudin risus tincidunt non.

    Etiam sit amet tempus lorem, aliquet condimentum velit. Donec et nibh
    consequat, sagittis ex eget, dictum orci. Etiam quis semper ante. Ut eu
    mauris purus. Proin nec consectetur ligula. Mauris pretium molestie
    ullamcorper. Integer nisi neque, aliquet et odio non, sagittis porta justo.

    \begin{itemize}
      \item \textbf{Sed consequat} id ante vel efficitur. Praesent congue massa
        sed est scelerisque, elementum mollis augue iaculis.
        \begin{itemize}
          \item In sed est finibus, vulputate
            nunc gravida, pulvinar lorem. In maximus nunc dolor, sed auctor eros
            porttitor quis.
          \item Fusce ornare dignissim nisi. Nam sit amet risus vel lacus
            tempor tincidunt eu a arcu.
          \item Donec rhoncus vestibulum erat, quis aliquam leo
            gravida egestas.
        \end{itemize}
      \item \textbf{Sed luctus, elit sit amet} dictum maximus, diam dolor
        faucibus purus, sed lobortis justo erat id turpis.
      \item \textbf{Pellentesque facilisis dolor in leo} bibendum congue.
        Maecenas congue finibus justo, vitae eleifend urna facilisis at.
    \end{itemize}

  \end{block}

\end{column}

\separatorcolumn

\begin{column}{\colwidth}

  \begin{block}{QIIME 2 analyses are interface-agnostic. Automatic provenance tracking saves time and alleviates uncertainty.}

    The body of our analysis will take multiple paths, incorporating data
    exploration, decision-making, and final analyses based on these. We will
    gloss over a number of steps in this discussion, allowing us to focus on
    key points. Complete method details of our analysis are available in the
    provenance graphs at \code{[QR CODE]: repo data outcomes.}

    After creating visual summaries of our denoised data and building a
    phylogenetic tree, we will investigate alpha and beta diversity. To use
    alpha and beta diversity, we must first subsample our samples to create a
    fair comparison. We call this process rarefaction.

    By manipulating a \code{[rich, interactive visualization]} in QIIME 2 View,
    we are able to select an appropriate “rarefaction depth” at which to
    subsample before rarefying with \code{core-metrics-phylogenetic}.

    \textbf{Using QIIME 2’s artifact API in a jupyter notebook gives us access to the
    full power of python 3, and allows us to iterate over sections of analysis
    independently, making quick changes to method parameters while preserving
    unaffected portions of our workflow.}

    We use this approach to experiment with different rarefaction depths,
    selecting the depth that best balances the cost of sample loss against the
    benefit of increased numbers of sequences per sample.

    \code{Jupyter Notebook: core-metrics-phylogenetic (and remaining alpha and beta diversity analysis?)}

    Once we’ve finished diversity analysis, we move our workflow into QIIME 2
    Studio (q2studio), a graphical user interface that provides convenient
    access to all of the QIIME 2 plugins in our environment, and also gives us
    \textbf{asynchronous process handling}. This gives us the ability to continue
    working while a computational process is running in the background.

    \textbf{Throughout this complex series of decisions and computations, QIIME 2
    handles all of the “methods” note-taking for us, and the provenance
    information included in all of our artifacts and visualizations alleviates
    the risk of data loss due to uncertainty about which file is correct}.

  \end{block}

  \begin{block}{QIIME 2 visualizations can be shared and viewed without a software install, and contain complete provenance information.}

    As we close in on reaching a conclusion and publishing our findings,
    interactive visualizations play a significant role in understanding and
    communicating about our data. \code{interactive visualizations from the
    analysis: e.g. taxa-barplots, etc}

    QIIME 2’s integrated and automatic tracking of data provenance allows us
    to review which methods and parameters were used in our analysis,
    simplifying production of a complete and accurate methods section and/or
    supplemental digital materials for submission.

    \textbf{By packaging data with provenance, QIIME 2 reduces the risk that the
    methods used to produce a given outcome are accidentally misreported}.

    \textbf{QIIME 2 has integrated citation information within the plugins, allowing
    researchers to export citations for publication. BibTeX citations for this
    poster were generated by \code{qiime tools citations}}.

    Going into the review process, we have a high level of confidence that our
    reported methods are accurate, that we have selected the visualizations
    that represent our final experimental methods, and that our citations for
    those methods are appropriate and complete.

    \textbf{Without the need to download any special software, members of a review
    committee can view rich, interactive visualizations, and inspect the
    methods used to produce a given data set or visualization.} Reviewers have
    a new tool for spot-checking that methods and outcomes are aligned.

    \code{q2view viz QR code: give it a try}

    Readers of published pieces are given insight into the analytical
    processes and parameters used, simplifying the process of crafting follow
    up studies which reproduce or extend the original research.

    \begin{table}
      \centering
      \begin{tabular}{l r r c}
        \toprule
        \textbf{First column} & \textbf{Second column} & \textbf{Third column} & \textbf{Fourth} \\
        \midrule
        Foo & 13.37 & 384,394 & \alpha \\
        Bar & 2.17 & 1,392 & \beta \\
        Baz & 3.14 & 83,742 & \delta \\
        Qux & 7.59 & 974 & \gamma \\
        \bottomrule
      \end{tabular}
      \caption{A table caption.}
    \end{table}

    Donec quis posuere ligula. Nunc feugiat elit a mi malesuada consequat. Sed
    imperdiet augue ac nibh aliquet tristique. Aenean eu tortor vulputate,
    eleifend lorem in, dictum urna. Proin auctor ante in augue tincidunt
    tempor. Proin pellentesque vulputate odio, ac gravida nulla posuere
    efficitur. Aenean at velit vel dolor blandit molestie. Mauris laoreet
    commodo quam, non luctus nibh ullamcorper in. Class aptent taciti sociosqu
    ad litora torquent per conubia nostra, per inceptos himenaeos.

    Nulla varius finibus volutpat. Mauris molestie lorem tincidunt, iaculis
    libero at, gravida ante. Phasellus at felis eu neque suscipit suscipit.
    Integer ullamcorper, dui nec pretium ornare, urna dolor consequat libero,
    in feugiat elit lorem euismod lacus. Pellentesque sit amet dolor mollis,
    auctor urna non, tempus sem.

  \end{block}

  \begin{block}{References}

    \nocite{*}
    \footnotesize{\bibliographystyle{plain}\bibliography{poster}}

  \end{block}

\end{column}

\separatorcolumn
\end{columns}
\end{frame}

\end{document}
